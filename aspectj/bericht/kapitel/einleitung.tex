\chapter{Einleitung}
\label{chap:einleitung}

Dieses Dokument ist der schriftliche Teil des Modules Informatikseminar an der Berner Fachhochschule. In den kommenden Kapiteln wird die Aspektorientierte Programmierung mit AspectJ vorgestellt und erkl�rt.

\nocite{laddad:aspectj}

\section{Auftrag}
\label{sec:einleitung_auftrag}

Der Auftrag ist es die folgenden Fragen mit diesem Bericht zu beantworten.

\textit{"Was versteht man unter dem Konzept der Aspektorientierten Programmierung?\\
Worin besteht der Vorteil gegen�ber der OOP? \\
Erl�utern Sie die wichtigsten Methoden und Ideen von AspectJ und stellen Sie heraus, in welcher Form OOP erweitert wird."} \cite{moodle:auftrag}

Unsere Erkenntnisse werden in diesem Dokument festgehalten. Anschliessend an die Abgabe dieses Berichtes erfolgt eine Pr�sentation im Plenum mit Fragerunde und Diskussion. 

In der ersten Besprechung mit dem betreuenden Dozenten wurde uns nahegelegt auf eine zu technische und detailreiche Ausarbeitung des Themas zu verzichten und stattdessen den Fokus auf die unterliegenden Konzepte und Vorteile der Aspektorientierten Programmierung zu legen insbesondere in der Pr�sentation.

\section{Vorgehen}
\label{sec:einleitung_vorgehen}

In einem ersten Schritt musste das notwendige Wissen aufgebaut und gefestigt werden. Dazu wurden verschiedenste Informationsquellen konsultiert. Als eine wichtige Basis dieses Berichtes dient jedoch das Buch \glqq AspectJ in Action\grqq \cite{laddad:aspectj}. Nach gemeinsamer Ausarbeitung der Struktur unseres Berichtes teilten wir die Kapitel auf und arbeiteten individuell weiter. \\
Durch Gegenlesen der vom Partner verfassten Abschnitten gelang es uns Fehler zu erkennen und einige Themen verst�ndlicher zu formulieren. Die Pr�sentation basiert auf dem Bericht, der Fokus liegt jedoch auf dem Kapitel \nameref{chap:aop}.