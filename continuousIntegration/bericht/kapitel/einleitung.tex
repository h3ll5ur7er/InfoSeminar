\chapter{Introduction}
\label{chap:introduction}
Ce document est la partie écrite du module Séminaire Informatique de l'Haute école spécialisée de Berne.

\section{Mission et Abstract}
\label{sec:intro_mission}
L'objectif de ce rapport est d'offrir aux lecteurs un aperçu de l'intégration continue (Continuous Integration) et des solutions existantes.

Dans une première partie la notion Intégration Continue et les concept correspondants seront expliqué (Tests Continue, Intégration de base de données continue, etc). De plus les meilleures pratiques seront présenté et les bénéfices qu'on reçoit si on implémente les concepts et respecte les meilleures pratiques.

Dans la deuxième partie du rapport on vous donneras une vue d'ensemble de touts les outils disponible pour pratiquer l'IC. À cause du nombre immense de différents outils, il ne nous sera pas possible de considérer tous les composant et fournisseurs existant. Le but est de démontrer les avantages et désavantages de quelques outils sélectionné, entre autres les outil les plus répandu. Comme serveurs de l'intégration continue Jenkins, TeamCity, Travis et TeamFoundationServer ont était choisit.

\section{Approche}
\label{sec:intro_approche}

Pour commencer, la connaissance de la matière devait être acquis et solidifiée. Dans notre parcours professionnel on avait déjà rencontré des systèmes de l'intégration continue, mais seulement comme utilisateurs et jamais comme administrateur. Après avoir définie la structure de notre rapport on a partagé les travaux et continué à travailler individuellement. 

Pour être capable de donner une évaluation des serveurs d'IC choisit et mieux les connaitre, on a décidé d'installer, configurer et tester chacun. Pour ces testes on a créé des projets très simples en C++ et Java avec des testes unitaires. Ces projets ont était intégré par les serveurs IC.

Après avoir finit la partie écrite, on a relu et corrigé le rapport ensemble.


