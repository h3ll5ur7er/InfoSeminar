\chapter{Introduction}
\label{chap:introduction}
Ce document est la partie écrite du module Séminaire Informatique de l'Haute école specialisée de Berne.

\section{Mission}
\label{sec:intro_mission}
L'objectif de ce rapport est d'offrir un aperçu de l'intégration continue (Continuous Integration) et des solutions existantes aux lecteur.

Dans une première partie la notion Intégration Continue et les concept correspondants seront expliquer. De plus il faut absolument mentionner les meilleures pratiques de l'IC et les bénéfices qu'on reçoit si on décide d'implémenter les concepts et respecte ces pratiques.

Dans une deuxième partie du rapport on vous donneras une vue d'ensemble de touts les outils disponible pour pratiquer l'IC. à cause du nombre immense de différent outils, il ne nous sera pas possible de considérer tous les composant et fournisseurs existant. Le but est de démontrer les avantages et désavantages de quelques outils sélectionné, entre autres les outil les plus répandu.

\section{Approche}
\label{sec:intro_approche}

Pour commencer, la connaissance de la matière devait être acquis et solidifiée. Dans notre parcours professionnel on avait déjà rencontré des système de l'intégration Continue, mais seulement comme utilisateurs et jamais comme administrateur. Après avoir définie la structure de notre rapport on a partagé les travaux et continué à travailler individuellement. Après avoir finit la partie écrite on a corrigé le travail ensemble.

Pour être capable de démontrer des différents serveurs de CI et mieux donner une évaluation, on a décidé de configurer et installer trois serveur en nuage. De plus on a créé on projet de teste en java et c\# pour illustrer un processus d'IC complet.
