\chapter{Introduction}
\label{chap:introduction}
Ce document est la partie écrite du module Séminaire Informatique de l'Haute école specialisée de Berne.

\section{Mission}
\label{sec:intro_mission}
L'objectif de ce rapport est d'offrir un apercu de l'intégration continue (Continuous Integration) et des solutions existantes aux lecteur.

Dans une première partie la notion Intégration Continue et les concept correspondants seront expliquer. De plus il faut absolument mentionner les meuilleures pratiques de l'IC et les benefices qu'on recoit si on decide d'implémenter ces concepts.

Dans une deuxième partie du rapport on vous donneras une vue d'ensemble de tout les outils disponible pour pratiquer l'IC. à cause du nombre immense de different outils, il ne nous sera pas possible de considérer tous les composant existant. Le but est de démontrer les avantages et désavantages des quelques outils sélectionné, entre autres les outil les plus répandu. 

\section{Approche}
\label{sec:intro_approche}

Pour commencer la connaissance de la matière devait être acquisée et solidifiée. Dans notre parcours professionnel on a déjà encontrer des système de l'Intégration Continue, mais seulement comme utilisateurs et jamais comme administrateur.
Après avoir definie la structure de notre rapport on a partagé les travaux et continué à travailler individuellement.

... à la fin (gegenlesen etc.)

Pour être capable à démontrer des differents serveurs de CI et mieux donner une évaluation, on a decidé de configurer et installer trois serveur en nuage. De plus on a créeé on projet de teste en java et c\# pour illustrer un processus d'IC complet.
