\chapter{Évaluation}
\label{eval}

\section{Logiciels de construction}

\subsection{Ant}

\subsection{Maven}

\section{Serveur de l'intégration continue}

\subsection{Definition des critères}

En dessous vous trouvez les critères avec une bref déscription qui seront utilisé pour evaluer les quatres serveurs de l'intégration continue qui ont été choisit pour ce travail. \footnote{\cite{ibmciserver}}

\begin{enumerate}
\item \textbf{Caractéristiques du produit} \\
\textit{Les charactéristiques du produit sont l'aspect le plus important quand on choisit un serveur d'IC. On doit savoir les éxigences qu'une entreprise a et de ce point de vue selectionner un logiciel.}
	\begin{itemize}
		\item Intégration avec des outils de gestion des versions \\
		\textit{Est l'outil que nous utilisons supporté? Quelles outils sont supporté?}
		\item Intégration avec l'outil de construction \\
		\textit{Est notre langage de programmation (compilateur) et notre outil de construction supporté?}
		\item Information en retour \\
		\textit{Quelles méthodes de l'information en retour existe et sont ils suffisant pour nous?}
		\item Labeling \\
		\textit{Est-il possible de donner des identifiers à des versions d'un logiciel?}
		\item Extensibilité \\
		\textit{Est-il possible d'écrire des extension propre pour le serveur si necessaire?}
	\end{itemize}
\item \textbf{Générale}
	\begin{itemize}
		\item Fiabilité et longévité
		\item Environnement cible
		\item Infrastructure
		\item Couts
		\item Type de logiciel
	\end{itemize}
\item \textbf{Taille de la communauté}
	\begin{itemize}
		\item Nombre d'utilisateurs
		\item Nombre de plug-ins
	\end{itemize}
\item \textbf{Utilisation}
	\begin{itemize}
		\item Facilité d'utilisation
		\item Complexité de l'installation
	\end{itemize}
\end{enumerate}
\newpage
\begin{landscape}
\subsection{Apercu des resultats}
\begin{table}[H]
	\centering
		\begin{tabular}{lp{4cm}p{4cm}p{4cm}p{4cm}} \toprule
			\textbf{Critères} & \href{https://jenkins-ci.org}{\textbf{Jenkins}} & \href{https://www.jetbrains.com/teamcity/}{\textbf{TeamCity}} & \href{https://travis-ci.org}{\textbf{Travis CI}} & \href{https://www.visualstudio.com/en-us/products/tfs-overview-vs.aspx}{\textbf{Team Foundation Server}} \\ \midrule
			\rowcolor{GrayRow}\textbf{Caractéristique du produit} &  &  &  &  \\ \midrule[0.16em]
			Outils de gestion des versions & Subversion/CVS(+plugins) & Subversion/CVS(+plugins) & github/Git & Git/TFVC \\ \midrule
			Outils de construction & & ++ (CLI) & + & \\ \midrule
			Information en retour & & + & ++ & \\ \midrule
			Labeling & & \checkmark & x & \\ \midrule
			Extensibilité & ++ & + & - & -- \\ \midrule
			\rowcolor{GrayRow}\textbf{Générale} &  &  &  &  \\ \midrule[0.16em]
			Fiabilité et longévité & \checkmark & \checkmark & \checkmark & \checkmark \\ \midrule
			Environnement cible & tous & tous & Linux & Microsoft Windows \\ \midrule
			Infrastructure & On-premises & On-premises & On-premises/SaaS & On-premises/SaaS \\ \midrule
			Couts & gratuit & Freemium* & Freemium* & Freemium \\ \midrule
			Type de logiciel & Open Source (MIT) & Propriétaire & Open Source (MIT) & Propriétaire \\ \midrule
			\rowcolor{GrayRow}\textbf{Taille de la communauté} & & & & \\ \midrule[0.16em]
			Nombre d'utilisateurs & many & 30'000 clients & 240'000 projets & many \\ \midrule
			Nombre de plugins & ++ & + & x & - \\ \midrule
			\rowcolor{GrayRow}\textbf{Utilisation} &  &  &  &  \\ \midrule[0.16em]
			Facilité d'utilisation & many & + & ++ & many \\ \midrule
			Complexité de l'installation & many & + & ++ & many \\
			\bottomrule[0.16em]
		\end{tabular}
	\caption{Serveurs de l'IC}
	\label{tab:serveurs_eval}
\end{table}
Freemium = C'est gratuit pour la version base, mais ca coute pour des editions plus grande (entreprise).\\
* free for open source projects
\footnote{\citep{jenkinsplugins} \citep{teamcityenv} \citep{tfsversioncontrol}}

\end{landscape}
\newpage

\subsection{Jenkins}
\clearpage
\subsection{TeamCity}
\begin{wrapfigure}{r}{0.2\textwidth}
  \begin{center}
    \includegraphics[width=0.18\textwidth]{bilder/teamcity512}
  \end{center}
  \caption{TeamCity Logo}
\end{wrapfigure}
\paragraph{Characteristique du produit} TeamCity est un serveur de l'intégration continue de JetBrains, les fabricants d'une grande nombre d'outil de developpment (comme IntelliJ). Il est optimisé pour construire des projets de Java ou .NET, mais supporte aussi Python, Ruby et beaucoup d'autre langage avec des plugins. De plus il existe l'option de travailler avec la ligne de commande. TeamCity est très adaptable et peut être individualiser extensivement. \\
Tous les configurations et tous l'utilisation est effectué par l'interface web. Comme voies d'information en retour TeamCity supporte des émails, des messages Jabber ou directement dans l'IDE. De plus TeamCity supporte des "Build Tag" pour identifier des constructions. Si la functionalité de TeamCity ne suffice pas, il est facilement possible d'écrire un plugin.

\paragraph{Générale} L'infrastructure pour soutenir TeamCity doit être mis en place par l'entreprise. Il existe trois options de licence de TeamCity.
\begin{itemize}
	\item TeamCity Professional (20 configurations et 3 agent de construction)
	\item TeamCity Enterprise (3-100 agent packs)
	\item TeamCity Additional Build Agent (+ 1 agent de construction)
\end{itemize}
La licence TeamCity Professional est gratuite mais limité. Pour les versions TeamCity Enterprise le nombre de projets est illimité, mais on peu incrementer le nombre d'agent de construction pour atteindre une meilleure pérformance quand il y a beaucoup de processus de construction en même temps. Pour des projéts Open Source TeamCity est gratuite, pour des jeunes pousses il y a des rabais.\footnote{\citep{teamcitybuy}}
\paragraph{Taille de la communauté}
JetBrains affirme que 30'000+ clients utilise TeamCity pour executer l'intégration continue (Boeing, HP...). Il y a une grande nombre de plugins disponible sur le page web de TeamCity.\footnote{\citep{teamcityplugins}}
\paragraph{Utilisation}
L'installation de TeamCity est assez facile. Il existe des archive des fichiers exprès pour des installations rapide. TeamCity est basé sur Java et utilise un serveur Tomcat. Pour l'installation rapide, tous ce qu'on devait faire est télécharger et extraire l'archive, et puis executer un script et completer la configuration intiale. Pour un système productive une installation un peu plus complexe est prévu.\\
En utilisant TeamCity en premier on est confondu par la grande nombre d'options de configuration. Mais quand-même il était possible et pas très difficile de construire et mettre en rapport notre environment teste. TeamCity est très volumineux, mais bien structuré et agréable pour l'utilisateur. Le faite qu'on peut faire toute la configuration sur l'interface web est un grande plus.
\begin{figure}[H]
	\centering
		\includegraphics[scale=0.35]{bilder/teamcityadmin}
	\caption{TeamCity intérface d'administration}
	\label{fig:travisgui}
\end{figure}
\clearpage
\subsection{Travis CI}
\begin{wrapfigure}{r}{0.4\textwidth}
  \begin{center}
    \includegraphics[width=0.38\textwidth]{bilder/Travis-CI-logo}
  \end{center}
  \caption{Travis CI Logo}
\end{wrapfigure}
\paragraph{Characteristique du produit}
Travis CI est un serveur de l'intégration continue très facile à mettre en service et avec une intégration excellente avec \href{https://github.com/}{github.com}. Il supporte beaucoup de différent langage et outil de construction. La liste complète peuvent être trouver dans la documentation\footnote{\citep{traviscidocs}}. Mais il seulement supporte git comme outil de gestion des versions.\\
De plus il offre multiple voies d'information en retour. Le plus facile est par email, mais il y a aussi la possibilité d'envoyer des messages par IRC, Slack, HipChat etc.\footnote{\citep{traviscinotification}}. Chaque construction recoit un identificateur numerique, mais il n'est pas possible de le changer. Il y a une API pour accéder à Travis (SaaS), mais l'extensibilité semble limité.

\paragraph{Générale}
Il existe trois version de Travis CI.
\begin{itemize}
	\item \href{https://travis-ci.org/}{Travis CI for Open Source}
	\item \href{https://travis-ci.com/}{Travis Pro}
	\item \href{https://enterprise.travis-ci.com/}{Travis Enterprise}
\end{itemize}
Les deux première versions sont accessible comme SaaS. Une version est pour des projets Open Source qui est gratuite et l'autre est pour des projets avec un dépot de github privée qui coute. La troisième version est pour des entreprises qui veulent mettre l'infrastructure comme des serveurs à disposition eux-mêmes (Linux).

\paragraph{Taille de la communauté}
Travis CI affirme sur la page web qu'il y a 246'506 projets Open Source qui sont testé et intégré sur leur platform. Sur la nombre d'utilisateurs des deux versions commercial il n'y a pas d'information.

\paragraph{Utilisation}
L'utilisation de Travis CI est très pratiques est facile. Si on a déja un dépot sur github, il faut seulement trois pas pour lancer l'intégration continue avec Travis.

\begin{enumerate}
	\item Login avec le compte de github sur travis
	\item Choisir le dépot
	\item Écrire un fichier .travis.yml pour définir la configuration
\end{enumerate}

Après ça chaque fois qu'il y a un changement sur le dépot les testes et la construction seras executé automatiquement. Pour construire notre projet de test en java (maven), d'executer les testes avec trois différent version de java et envoyer un email à une adresse si quelque chose ne marche pas, le fichier en bas suffisait. De plus vous trouver un apercu de l'interface d'administration de travis.

\begin{figure}[H]
	\centering
		\includegraphics[scale=0.2]{bilder/travisciymlfile}
	\caption{Travis intérface d'administration et yml fichier}
	\label{fig:travisgui}
\end{figure}
\clearpage
\subsection{Team Foundation Server}



