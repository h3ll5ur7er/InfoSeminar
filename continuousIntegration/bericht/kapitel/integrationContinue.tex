\chapter{Intégration Continue}
\label{chap:integrationcontinue}

\nocite{duvall:conint}

\section{Aperçu}

Le processus de developpement d'un logiciel.
Ce que c'est CI, la definition?
Les concepts core, quels composant faut-il qu'on parle de CI?

\section{Concepts}

\subsection{Construction continue}

\subsection{Intégration continue de base de donnée}

\subsection{Test continue}

\subsection{Inspection continue}

\subsection{Déploiment continue}

\subsection{Information en retour continue}

\section{Motivation}

\subsection{Éviter des risques}

\begin{itemize}
\item Déploiement du logiciel
\item Découverte tarde des erreurs
\item Manque de visibilité du projet
\item Basse qualité de logiciel
\end{itemize}

\section{Meuilleures pratiques}

Peut-être c'est mieux de ne pas faire une intégration continue complet dans tous les cas. Pas introduire tout les concepts en meme temps, seulement si necessaire.
Meuilleures pratiques

\paragraph{Commit code frequently}

\paragraph{Dont commit broken code}

\paragraph{Fix broken builds immediately}

\paragraph{Write automated developer tests}

\paragraph{All tests and inspections must pass}

\paragraph{Run private builds}

\paragraph{Avoid getting broken code}