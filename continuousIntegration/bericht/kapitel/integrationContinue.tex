\chapter{Intégration Continue}
\label{chap:integrationcontinue}

\nocite{duvall:conint}

\section{Aperçu}

\subsection{Motivation}

\section{Concepts}

\section{Meilleures Pratiques}